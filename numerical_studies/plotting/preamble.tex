\makeatletter
\DeclareOldFontCommand{\rm}{\normalfont\rmfamily}{\mathrm}
\DeclareOldFontCommand{\sf}{\normalfont\sffamily}{\mathsf}
\DeclareOldFontCommand{\tt}{\normalfont\ttfamily}{\mathtt}
\DeclareOldFontCommand{\bf}{\normalfont\bfseries}{\mathbf}
\DeclareOldFontCommand{\it}{\normalfont\itshape}{\mathit}
\DeclareOldFontCommand{\sl}{\normalfont\slshape}{\@nomath\sl}
\DeclareOldFontCommand{\sc}{\normalfont\scshape}{\@nomath\sc}
\makeatother

\usepackage{authblk}
\usepackage{subfiles}

\usepackage{subfigure}
\usepackage{xcolor}

\usepackage{physics}
\usepackage{textgreek}
\usepackage{algorithm}
\usepackage{algpseudocode}
\usepackage{nicefrac}
\usepackage[acronym]{glossaries}

\usepackage[obeyFinal]{todonotes}
\newcommand\todoin[2][]{\todo[inline, caption={2do}, #1]{\begin{minipage}{\textwidth-10pt}#2\end{minipage}}}
\usepackage{amsmath}
\newcommand\todoreviewhb[2][]{\todo[inline,color=blue!50,#1 author=HB]{#2}}
\newcommand\todoreviewbu[2][]{\todo[inline,color=blue!50,#1 author=BU]{#2}}
% for plotting
\usepackage{tikz}
\usetikzlibrary{decorations.pathmorphing,patterns, arrows, calc}
\usepackage{pgfplots}
\pgfplotsset{compat = newest}

\usepackage{pgfplots}
\pgfkeys{/pgfplots/tuftelike/.style={
		semithick,
		tick style={major tick length=4pt,semithick,black},
		separate axis lines,
		axis x line*=bottom,
		axis x line shift=10pt,
		axis y line*=left,
		axis y line shift=10pt}}

\ifx\pdfoutput\undefined
\usepackage[hypertex]{hyperref}
\else
\usepackage[pdftex,hypertexnames=false]{hyperref}
\fi
\newcommand{\algorithmautorefname}{Algorithm}  % needed for correct naming of algorithms
\usepackage[noabbrev,nameinlink]{cleveref}
\crefname{subsection}{subsection}{subsections} % needed for correct naming of subsections

\definecolor{color1}{HTML}{008b8b}
\definecolor{color2}{HTML}{808000}

% for nicer plots
\usepackage{pgfplots}
\pgfkeys{/pgfplots/tuftelike/.style={
  semithick,
  tick style={major tick length=4pt,semithick,black},
  separate axis lines,
  axis x line*=bottom,
  axis x line shift=10pt,
  axis y line*=left,
  axis y line shift=10pt}}

% For properly referencing subfigures
\newcommand{\subfigureautorefname}{\figureautorefname}

%%%%%%%%%%%%%%%%%%%%%%%%%%%%%%%%%%%%%%%%%%%%%%%%%%%%%%%%%%%%%%%%%%%%%%%%%%
% 	Acronyms
%%%%%%%%%%%%%%%%%%%%%%%%%%%%%%%%%%%%%%%%%%%%%%%%%%%%%%%%%%%%%%%%%%%%%%%%%%

\newacronym{ode}{ODE}{ordinary differential equation}

\newacronym{pde}{PDE}{partial differential equation}

\newacronym{dae}{DAE}{differential algebraic equation}

\newacronym{fsi}{FSI}{fluid-structure interaction}

\newacronym{cps}{CPS}{conventional parallel staggered}
\newacronym{css}{CSS}{conventional serial staggered}